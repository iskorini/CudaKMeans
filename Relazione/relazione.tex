\documentclass[10pt,twocolumn,letterpaper]{article}

\usepackage{cvpr}
\usepackage{times}
\usepackage{epsfig}
\usepackage{graphicx}
\usepackage{amsmath}
\usepackage{amssymb}

\usepackage{url}
\usepackage{algorithm}% http://ctan.org/pkg/algorithms
\usepackage{algpseudocode}% http://ctan.org/pkg/algorithmicx

% Include other packages here, before hyperref.

% If you comment hyperref and then uncomment it, you should delete
% egpaper.aux before re-running latex.  (Or just hit 'q' on the first latex
% run, let it finish, and you should be clear).
\usepackage[pagebackref=true,breaklinks=true,letterpaper=true,colorlinks,bookmarks=false]{hyperref}

\cvprfinalcopy % *** Uncomment this line for the final submission


\def\httilde{\mbox{\tt\raisebox{-.5ex}{\symbol{126}}}}

% Pages are numbered in submission mode, and unnumbered in camera-ready
\ifcvprfinal\pagestyle{empty}\fi



\makeatletter
\newcommand{\newalgname}[1]{%
  \renewcommand{\ALG@name}{#1}%
}
\newalgname{Algorithm}% All algorithms will be called "Algorithme"
\renewcommand{\listalgorithmname}{List of \ALG@name s}
\makeatother

\begin{document}

%%%%%%%%% TITLE
\title{PC-2016/17 Course Project \\
Implementation in CUDA of KMeans algorithm}

\author{Tommaso Ceccarini \\
E-mail address\\
{\tt\small tommaso.ceccarini1@stud.unifi.it}
% For a paper whose authors are all at the same institution,
% omit the following lines up until the closing ``}''.
% Additional authors and addresses can be added with ``\and'',
% just like the second author.
% To save space, use either the email address or home page, not both
\and
Federico Schipani\\
E-mail address\\
{\tt\small federico.schipani@stud.unifi.it}
}

\maketitle
\thispagestyle{empty}

%%%%%%%%% ABSTRACT
\begin{abstract}
   KMeans algorithm is one of the most popular method for clustering analysis.
   In our work we provide a CUDA implementation that use an Nvidia GPU to solve the clustering problem.
   We also provide a performance analysis with the purpose of compare the performance of our parallel CUDA implementation with a sequential implementation written in C language.
\end{abstract}

%%%%%%%%% BODY TEXT
\section{Introduction}
KMeans algorithm is one of the most popular method for clustering analysis.
The purpose of the cluster analysis is that of divide data into meaningfull group, called cluster.
The resultant cluster should then capture the structure of the data.\par
KMeans methods attempt to do this by evaluating a similarity measure according to the mean value of
the data that are cointained in the clusters.
So, given a set of observation $(\boldsymbol{x_{1}}, \boldsymbol{x_{2}}, \dots,\boldsymbol{x_{N}} )$ where each observation is a $P-dimensional$ real vector, k-means clustering aims to partition the $N$ observation into $K (\leq N)$ sets $\boldsymbol{S} = \{S_{1}, S_{2}, \dots, S_{K}\}$ so as to minimize the within-cluster sum of squares. In other words, its objective is to find: 
\begin{equation}
\label{eq:first}
\operatorname*{arg\, min}_{\boldsymbol{S}} \displaystyle\sum_{i = 1}^{k} \displaystyle\sum_{x \in S_{i}} ||\boldsymbol{x} - \boldsymbol{\mu}_{i} ||^{2}
\end{equation}  
where $\mu_i$ is the mean of points in $S_{i}$.\cite{wiki:kmeans}
\par
\begin{algorithm}
\label{alg:kmeans}
\caption{KMeans}
\begin{algorithmic}[1]
\Procedure{KMeansSequential}{$\text{data}[\text{n}][\text{p}]$}
\State $\text{mean}[\text{k}][\text{p}], \text{oldMean}[\text{k}][\text{p}]$
\State $\text{assignment}[\text{n}]$
\State $(\text{mean}, \text{assignment}) \leftarrow initAss(data)$
\While{$!stop$}
\State $\text{assignment} \leftarrow calcMin(\text{mean}, \text{data})$
\State $\text{oldMean} \leftarrow \text{mean}$
\State $\text{mean} \leftarrow calcMean(\text{assignment}, \text{data})$
\State $\text{stop} \leftarrow stopCriterion(\text{mean}, \text{oldMean})$
\EndWhile
\State \textbf{return} $\text{mean}$
\EndProcedure
\end{algorithmic}
\end{algorithm}
Algorithm \ref{alg:kmeans} is the pseudocode of an iterative implementation of KMeans algorithm.
Before the \textit{while} cycle there are the initial assignment step. In this step the first $k$ data are assigned to the first $k$ cluster. A pseudocode of this initial assignment is showed in Algorithm \ref{alg:initialAssignment}

\begin{algorithm}
\label{alg:initialAssignment}
\caption{Initial Assignment}
\begin{algorithmic}[1]
\Procedure{initAss}{$\text{data}[\text{n}][\text{p}]$}
\State $\text{mean}[\text{k}][\text{p}]$
\State $\text{assignment}[\text{n}]$
\For{$\text{i} = 0; \text{i}<\text{k}; \text{i}++$}
\State $\text{assignment}[\text{i}] = \text{i}$
\For{$\text{j} = 0; \text{j} < \text{p}: \text{j}++$} 
\State $\text{mean}[\text{i}][\text{j}] = \text{data}[\text{i}][\text{j}]$
\EndFor
\EndFor
\State \textbf{return} $\text{(mean, assignment)}$
\EndProcedure
\end{algorithmic}
\end{algorithm}


\vspace{3cm}


% section sequential_kmeans (end)




\bibliographystyle{plain}
\bibliography{bib.bib}




\end{document}
